%%%%%
% Author:
% Ismani Nieuweboer
%   ismani.nieuweboer@student.uva.nl
%   10502815
%
% Compile from this file to create the pdf.
% Last checked working on overleaf.com
%%%%%

\documentclass{article}
\usepackage{structure/mystyle}

% Bibliografie
\usepackage{csquotes}
\usepackage{ragged2e}
\usepackage[
  % niks qua stijl enzo lijkt het beste
  backend=biber,
]{biblatex}
\addbibresource{refs.bib}

% Hyperref als bijna laatste
\usepackage{hyperref}
% \hypersetup{
%     colorlinks=true,
%     linkcolor={red!50!black},
%     citecolor={blue!50!black},
%     urlcolor={blue!80!black},
% }
% \hypersetup{
%     colorlinks=true,
%     linkcolor={black},
%     citecolor={black},
%     urlcolor={blue!80!black},
% }
\hypersetup{
    colorlinks=true,
    linkcolor={black},
    citecolor={blue!50!black},
    urlcolor={blue!80!black},
}
%\usepackage{fncylab}

% \addto\extrasenglish{\def\chapterautorefname{chapter}}
% \addto\extrasenglish{\def\sectionautorefname{section}}
% \usepackage[nameinlink, noabbrev]{cleveref}  % lowercase




\title{Econophysics and wealth distribution models}
\author{Ismani Nieuweboer}
% \date{September 2018}

\begin{document}

\maketitle

% \url{https://arxiv.org/abs/0905.1518} %yakovenko
% \url{https://arxiv.org/abs/1309.3916} %redig

% not directly related, from sociophysics
% \url{https://arxiv.org/abs/1802.07068} % Talent vs Luck: the role of randomness in success and failure
% \url{https://arxiv.org/abs/1811.05206} %Exploring the role of talent and luck in getting success

% https://en.yaronshemesh.com/inequality/
% https://smus.com/simulating-wealth-inequality/
% http://borismus.github.io/inequality-simulator/?model=1-world-income-ineq-doesnt-lead-to-wealth-ineq.js


%further reading by Redig
% \url{https://arxiv.org/abs/1508.04918}
% \url{https://arxiv.org/abs/1606.08692}






\section*{Introduction}
Wealth distributions ...

The topic arises in a context of interacting particle systems, described in detail in  \cite{liggett2012interacting, liggett2013stochastic}. For an overview of applications of interacting particle systems see also the list given in \cite{frankredig2014}.



\section{Preliminaries}


\section{Product measures}

Under redistribution model with kernel $\mu$,
\[
\mu(x, y) \diff x \diff y  \text{ invariant iff }
\nu(a, s) = \frac{ \mu(as, (1-a)s) }{\int_0^1 \mu(\alpha s, (1-\alpha)s) \diff \alpha}
\]

\section{Stationary for wealth dist}

\[
Lf(r) = \int \paren[\big]{f(\lambda r + (1 - \lambda)\epsilon) - f(r) } \diff \nu(\epsilon)
\]
must be stationary under iteration
\[
R_{n+1} = \lambda r_n + (1-\lambda)\epsilon_n
\\= \lambda^{n+1} + (1-\lambda)\sum_{k=0}^n \lambda^{k} \epsilon_{n-k}
\\\overset{\diff}{=} \lambda^{n+1} + (1-\lambda)\sum_{k=0}^n \lambda^{k} \epsilon_{k}
\to 
 (1-\lambda)\sum_{k=0}^{\infty} \lambda^{k} \epsilon_{k}
\]
which gives us the unique stationary distribution for $L$ $\nu_{\infty}$



\section{Duality}
The following theorem is adapted from \cite{barbour2000transition}.

Let $(X_t)_t$ be a Markov process with Markov generator $L$ on some continuous state space $\Omega$, with some invariant measure $\mu$. Let $f_{\eta} \colon \Omega \to \Omega$. Assume
\[
L f_{\eta} = \sum_{\zeta} r(\eta, \zeta) f_{\zeta}
\]
for rates $r(\eta, \zeta) \ge 0$ if $\eta \ne \zeta$ and $r(\eta, \eta) \le 0$ for all $\eta$.

Define a generator $Q$ on a discrete state space $\Omega'$ by
\[
q(\eta, \zeta) \defeq r(\eta, \zeta)\frac{\mu(f_{\zeta})}{\mu(f_{\eta})}.
\]
Note that these indeed are rates as
\[
\sum_{\zeta} q(\eta, \zeta)
= \frac{1}{\mu(f_{\eta})} \sum_{\zeta} r(\eta, \zeta)\mu(f_{\zeta})
= \frac{1}{\mu(f_{\eta})} \mu\paren*{\sum_{\zeta} r(\eta, \zeta) f_{\zeta}}
= \frac{1}{\mu(f_{\eta})} \mu(L f_{\eta})
= 0.
\]
Let $(X_t')_t$ be the associated Markov process. We then have duality with the duality function
\[
D(\eta, x) = \frac{f_{\eta}}{\mu(f_{\eta})}
\]
as
\[
LD(\eta, \cdot)(x)
= \frac{f_{\eta}(x)}{\mu(f_{\eta})}
= \sum_{\zeta} \frac{r(\eta, \zeta)}{\mu(f_{\eta})} f_{\zeta}(x)
= \sum_{\zeta} \frac{q(\eta, \zeta)}{\mu(f_{\zeta})} f_{\zeta}(x)
= \sum_{\zeta} q(\eta, \zeta) D(\zeta, x)
= Q D(\cdot, x)(\eta)
\]
which implies $\exp(tL) D(\eta, \cdot)(x) = \exp(tQ) D(\cdot, x)(\eta)$ (the proof of this detail is technical, see \cite{voss2011equivalence} for a proof), which in turn implies (by definition) % https://www.researchgate.net/profile/Anja_Voss-Boehme/publication/261811929_On_the_Equivalence_Between_Liggett_Duality_of_Markov_Processes_and_the_Duality_Relation_Between_Their_Generators/links/00b7d5358d931ad3c6000000/On-the-Equivalence-Between-Liggett-Duality-of-Markov-Processes-and-the-Duality-Relation-Between-Their-Generators.pdf
\[
\ye{x}{D(\alpha, x_t}
= \ye{\alpha}{D(\alpha_t, x}.
\]


One can wonder whether there is a correspondence between duality of Markov processes and duality in the sense of dual spaces and adjoints of operators -- there happens to be one, explored in \cite{2012arXiv1210.7193J}. % https://arxiv.org/pdf/1210.7193.pdf




\section{Diffusion}

% http://www.math.wisc.edu/~shottovy/NumPDEreport.pdf
% https://www.math.nyu.edu/faculty/goodman/teaching/StochCalc2013/notes/Week9.pdf

Let $\mu(r) = \frac{a(rs, (1-r)s)}{s}, h(r) = \frac{\sigma(r)^2}{2}=(1-r)r$

Probability density should go to zero approaching the boundary points (in this case $0, 1$) 

The density corresponding to the stationary distribution of a process with generator
\[
(L_s^r \cdot)(r)
= \mu(r) \partial_r + \frac{\sigma(r)^2}{2} \partial_r^2
= \mu(r) \partial_r + h(r) \partial_r^2
\]
of a diffusion process satisfies the Kolmogorov forward equation (also called the Fokker--Planck equation): %, see also the Feynman--Kac equation corresponding to the backward equation):
\[
0 = -\brackets{g(r) \psi(r)}' + \brackets{h(r) \psi(r)}''
\\\implies g(r) \psi(r) = \brackets{h(r) \psi(r)}'
\]
due to boundary conditions. Setting $y(r) = h(r) \psi(r)$ we get
\[
y' = \frac{g}{h} y \implies (\ln(y))' = \frac{g}{h}
\\ \implies y(r) = C \exp\paren*{ \int \frac{g(r)}{h(r)}\diff r }
\\ \implies \psi(r) = \frac{C}{h(r)} \exp\paren{ \int \frac{g(r)}{h(r)}\diff r }
\]
i.e.
\[
\psi(r) = \frac{C}{r(1-r)} \exp\paren{ \int \frac{a(rs, (1-r)s)}{r(1-r)s}\diff r }
\]


\section*{Conclusion}
For the wealth distribution model discussed in ..., in the case of nonzero propensity i.e. if the agents save any money there do not exist any product stationary measures.



\printbibliography

\end{document}