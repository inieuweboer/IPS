%%%%%
% Author:
% Ismani Nieuweboer
%   ismani.nieuweboer@student.uva.nl
%   10502815
%
% Compile from this file to create the pdf.
% Last checked working on overleaf.com
%%%%%

\documentclass{article}
\usepackage{structure/mystyle}

% Bibliografie
\usepackage{csquotes}
\usepackage{ragged2e}
\usepackage[
    backend=bibtex,
    % bibencoding=ascii,  % FOR SOME REASON THIS MUST BE DISABLE
    % citestyle=authoryear 
    citestyle=authoryear-icomp
    % style=numeric,
    style=alphabetic,
    % sorting=none,
]{biblatex}



\title{Econophysics and wealth distribution models}
\author{Ismani Nieuweboer}
% \date{September 2018}

\begin{document}

\maketitle

\url{https://arxiv.org/abs/0905.1518}
\url{https://arxiv.org/abs/1309.3916}
\url{https://arxiv.org/abs/1802.07068}
\url{https://arxiv.org/abs/1811.05206}

http://www.math.wisc.edu/~shottovy/NumPDEreport.pdf

% https://en.yaronshemesh.com/inequality/
% https://smus.com/simulating-wealth-inequality/
% http://borismus.github.io/inequality-simulator/?model=1-world-income-ineq-doesnt-lead-to-wealth-ineq.js




\section*{Introduction}
Wealth distributions ...

The topic arises in a context of interacting particle systems, described by T. Liggett \ref{liggett2012interacting}\ref{liggett2013stochastic}. For an overview of applications of interacting particle systems see also the list given in \ref{frankredig2014}.



\section{Preliminaries}


\section{Product measures}

Under redistribution model with kernel $\mu$,
\[
\mu(x, y) \diff x \diff y  \text{ invariant iff }
\nu(a, s) = \frac{ \mu(as, (1-a)s) }{\int_0^1 \mu(\alpha s, (1-\alpha)s) \diff \alpha}
\]

\section{Stationary for wealth dist}

\[
Lf(r) = \int \paren[\big]{f(\lambda r + (1 - \lambda)\epsilon) - f(r) } \diff \nu(\epsilon)
\]
must be stationary under iteration
\[
R_{n+1} = \lambda r_n + (1-\lambda)\epsilon_n
\\= \lambda^{n+1} + (1-\lambda)\sum_{k=0}^n \lambda^{k} \epsilon_{n-k}
\\\overset{\diff}{=} \lambda^{n+1} + (1-\lambda)\sum_{k=0}^n \lambda^{k} \epsilon_{k}
\to 
 (1-\lambda)\sum_{k=0}^{\infty} \lambda^{k} \epsilon_{k}
\]
which gives us the unique stationary distribution for $L$ $\nu_{\infty}$

\hdots


\section{Duality}
% https://www.researchgate.net/profile/Anja_Voss-Boehme/publication/261811929_On_the_Equivalence_Between_Liggett_Duality_of_Markov_Processes_and_the_Duality_Relation_Between_Their_Generators/links/00b7d5358d931ad3c6000000/On-the-Equivalence-Between-Liggett-Duality-of-Markov-Processes-and-the-Duality-Relation-Between-Their-Generators.pdf



\section{Diffusion}

% https://www.math.nyu.edu/faculty/goodman/teaching/StochCalc2013/notes/Week9.pdf

Let $\mu(r) = \frac{a(rs, (1-r)s)}{s}, h(r) = \frac{\sigma(r)^2}{2}=(1-r)r$

Probability density should go to zero approaching the boundary points (in this case $0, 1$) 

The density corresponding to the stationary distribution of a process with generator
\[
(L_s^r \cdot)(r)
= \mu(r) \partial_r + \frac{\sigma(r)^2}{2} \partial_r^2
= \mu(r) \partial_r + h(r) \partial_r^2
\]
of a diffusion process satisfies the Kolmogorov forward equation (also called the Fokker--Planck equation, see also the Feynman--Kac equation corresponding to the backward equation):
\[
0 = -\brackets{g(r) \psi(r)}' + \brackets{h(r) \psi(r)}''
\\\implies g(r) \psi(r) = \brackets{h(r) \psi(r)}'
\]
due to boundary conditions. Setting $y(r) = h(r) \psi(r)$ we get
\[
y' = \frac{g}{h} y \implies (\ln(y))' = \frac{g}{h}
\\ \implies y(r) = C \exp\paren*{ \int \frac{g(r)}{h(r)}\diff r }
\\ \implies \psi(r) = \frac{C}{h(r)} \exp\paren{ \int \frac{g(r)}{h(r)}\diff r }
\]
i.e.
\[
\psi(r) = \frac{C}{r(1-r)} \exp\paren{ \int \frac{a(rs, (1-r)s)}{r(1-r)s}\diff r }
\]


\section*{Conclusion}
For the wealth distribution model discussed in ..., in the case of nonzero propensity i.e. if the agents save any money there do not exist any product stationary measures.




\end{document}