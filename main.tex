%%%%%
% Author:
% Ismani Nieuweboer
%   ismani.nieuweboer@student.uva.nl
%   10502815
%
% Compile from this file to create the pdf.
% Last checked working on overleaf.com
%%%%%

\documentclass{article}
\usepackage{structure/mystyle}

% Bibliografie
\usepackage{csquotes}
\usepackage{ragged2e}
\usepackage[
  % niks qua stijl enzo lijkt het beste
  backend=biber,
]{biblatex}
\addbibresource{refs.bib}

% Hyperref als bijna laatste
\usepackage{hyperref}
% \hypersetup{
%     colorlinks=true,
%     linkcolor={red!50!black},
%     citecolor={blue!50!black},
%     urlcolor={blue!80!black},
% }
% \hypersetup{
%     colorlinks=true,
%     linkcolor={black},
%     citecolor={black},
%     urlcolor={blue!80!black},
% }
\hypersetup{
    colorlinks=true,
    linkcolor={black},
    citecolor={blue!50!black},
    urlcolor={blue!80!black},
}
%\usepackage{fncylab}

% \addto\extrasenglish{\def\chapterautorefname{chapter}}
% \addto\extrasenglish{\def\sectionautorefname{section}}
% \usepackage[nameinlink, noabbrev]{cleveref}  % lowercase




\title{Stationary distributions and Markov duality of a wealth distribution model}
\author{Ismani Nieuweboer}
% \date{September 2018}

\begin{document}

\maketitle

% \url{https://arxiv.org/abs/0905.1518} %yakovenko
% \url{https://arxiv.org/abs/1309.3916} %redig

% not directly related, from sociophysics
% \url{https://arxiv.org/abs/1802.07068} % Talent vs Luck: the role of randomness in success and failure
% \url{https://arxiv.org/abs/1811.05206} %Exploring the role of talent and luck in getting success

% https://en.yaronshemesh.com/inequality/
% https://smus.com/simulating-wealth-inequality/
% http://borismus.github.io/inequality-simulator/?model=1-world-income-ineq-doesnt-lead-to-wealth-ineq.js


%further reading by Redig
% \url{https://arxiv.org/abs/1508.04918}
% \url{https://arxiv.org/abs/1606.08692}


% \nocite{*}




\section*{Introduction}
Wealth distributions ...

The topic arises in a context of interacting particle systems, described in detail in  \cite{liggett2012interacting, liggett2013stochastic}. For an overview of applications of interacting particle systems see also the list given in \cite{frankredig2014}.

For an historical overview of econophysics, see \cite{2008arXiv0802.1416D, 2011arXiv1108.0977S}

After discussing the preliminaries from \cite{frankredig2014} in \autoref{sec:Preliminaries}, ...


\section{Preliminaries}%\label{prelim}
Beta distribution

Define the Beta function as
\[
B(\alpha, \beta)
= \int_{0}^{1} t^{\alpha-1} (1-t)^{\beta-1} \diff t
\]
Define the Gamma function for $\operatorname*{Re}(z) > 0$:
\[
\Gamma(z)
= \int_{0}^{\infty} t^{z-1} \exp(-t) \diff t
\]
This is mostly known as the real and complex analytical continuation of the factorial function -- we have for a positive integer $n$ that $\Gamma(n) = (n-1)!$. Using this the function can even be defined for all $z \not \in \zz_{\le 0}$, although this will not be used.

We have the identity
\[
B(\alpha, \beta)
= \frac{\Gamma(\alpha) \Gamma(\beta)}{\Gamma(\alpha + \beta)}
\]

The Beta distribution is characterized by probability density $p \colon [0, 1] \to \rr_{\ge 0}$ such that
\[
p(x) \propto x^{\alpha - 1} (1-x)^{\beta - 1}.
\]
Normalizing gives
\[
p(x) = \frac{1}{B(\alpha, \beta)} x^{\alpha - 1} (1-x)^{\beta - 1}.
\]

Note that the uniform distribution is a special case of the Beta distribution with $\alpha = \beta = 1$.


The Gamma distribution is characterized by probability density $p \colon \rr_{>0} \to \rr_{\ge 0}$ such that
\[
p(x) \propto x^{\alpha - 1} \exp\paren{-\lambda x}.
\]
The parameters $\alpha, \lambda$ are called the \emph{shape} and \emph{rate} respectively. Normalizing gives
\[
p(x) = \frac{\lambda^{\alpha}}{\Gamma(\alpha)} x^{\alpha - 1} \exp\paren{-\lambda x}.
\]

For $X \sim \Gamma(\alpha, \lambda)$
\[
\xe{X^n}
= \frac{\Gamma(\alpha + n))}{\lambda^{\alpha} \Gamma(\alpha)}
\]


\section{Research questions
\begin{itemize}
\item What are the stationary distribution in this model?
\item Can we write the stationary distributions as a product?
\item Do there exist discrete dual processes (and if yes under which duality functions)?
\item Can anything be said for $N$ agents?
\end{itemize}

\section{Energy and wealth distribution model}


\section{Product measures}

Under redistribution model with kernel $\mu$,
\[
\mu(x, y) \diff x \diff y  \text{ invariant iff }
\nu(a, s) = \frac{ \mu(as, (1-a)s) }{\int_0^1 \mu(\alpha s, (1-\alpha)s) \diff \alpha}
\]

We prove this by showing that if $\mu$ is an invariant measure, then for all bounded continuous $f$
\[
\int_0^{\infty} \int_0^1 f(rs, (1-r)s) \mu(rs, (1-r)s) s \diff r \diff s
% = \int_0^{\infty} \int_0^1 f(rs, (1-r)s) \int_0^1 \mu(\epsilon s, (1-\epsilon) s) \nu(s, r) s \diff r \diff s;
&= \int_0^{\infty} \int_0^{1} f(r s, (1-r) s) \paren*{ \int_{0}^{1} \mu(\epsilon s, (1-\epsilon) s)\diff \epsilon} \nu(s, r)s \diff r \diff s;
\]
this implies immediately that
\[
\mu(r s, (1-r) s) = 
\int_0^1 \mu(\epsilon s, (1-\epsilon) s) \nu(s, r)
\implies
\nu(s, r)
\propto \mu(r s, (1-r) s).
\]
This is done by combining the following three (strings of) equations, with one application of Fubini:
\[
\mu(Pf)
&= \mu(f)
\\
\mu(f)
&= \int_0^{\infty} \int_0^{\infty} f(x, y) \mu(\diff x, \diff y)
= \int_0^{\infty} \int_0^{1} f(rs, (1-r)s) \mu(rs, (1-r)s) s \diff r \diff s
\\
\mu(Pf)
&= \int_0^{\infty} \int_0^{\infty} Pf(x, y) \mu(\diff x, \diff y)
\\&= \int_0^{\infty} \int_0^{\infty} \int_{0}^{1} f(\epsilon (x+y), (1-\epsilon) (x+y)) \nu(x+y, \diff\epsilon)  \mu(x, y) \diff x \diff y
\\&= \int_0^{\infty} \int_0^{1} \int_{0}^{1} f(\epsilon s, (1-\epsilon) s) \nu(s, \epsilon)\diff \epsilon  \mu(rs, (1-r) s) s \diff r \diff s
\\&= \int_0^{\infty} \int_0^{1} \int_{0}^{1} f(r s, (1-r) s) \nu(s, r)\diff r  \mu(\epsilon s, (1-\epsilon) s) s \diff \epsilon \diff s
\\&= \int_0^{\infty} \int_0^{1} f(r s, (1-r) s) \paren*{ \int_{0}^{1} \mu(\epsilon s, (1-\epsilon) s)\diff \epsilon} \nu(s, r)s \diff r \diff s.
\]
The other implication follows easily by combining these equations in the reverse direction.

Reversible:
\[
\mu(Pf \cdot g)
&= \int_0^{\infty} \int_0^{\infty} Pf(x, y) g(x, y) \mu(x, y) \diff x \diff y
\\&= \int_0^{\infty} \int_0^{\infty} \int_0^1 f(\epsilon (x+y), (1-\epsilon) (x+y)) \diff \epsilon g(x, y) \mu(x, y) \diff x \diff y
\\&= \int_0^{\infty} \int_0^{1} \int_0^1 f(\epsilon s, (1-\epsilon) s)  \diff \epsilon g(rs, (1-r)s) \mu(rs, (1-r) s) s \diff r \diff s
\\&= \int_0^{\infty} \int_0^{1} \int_0^1 f(r s, (1-r) s)  \diff r g(\epsilon s, (1-\epsilon)s) \mu(\epsilon s, (1-\epsilon)s) s \diff \epsilon \diff s
\\&= \int_0^{\infty} \int_0^{1} \int_0^1 g(\epsilon s, (1-\epsilon) s)  \diff \epsilon f(rs, (1-r)s) \mu(rs, (1-r) s) s \diff r \diff s
\\&= \mu(Pg \cdot f)
\]

Gamma:

Set $\psi(s) = \int_0^1 \mu(as) \mu((1-a) s) \diff a$. Since $\nu(r) = \nu(1-r)$ we can write $\nu(r) = K(r) K(1-r)$, giving
\[
K(r) K(1-r)
= \frac{\mu(rs) \mu((1-r) s)}{\psi(s)}
\implies
\psi(s)
= \frac{\mu(rs) \mu((1-r) s)}{K(r) K(1-r)}.
\]
Taking logarithms and then the derivative w.r.t. $r$ gives
\[
\ln\psi(s)
= \ln(\mu(rs)) + \ln(\mu((1-r) s)) - \ln(K(r)) - \ln(K(1-r))
\\\implies
0 = \frac{\mu'(rs) s}{\mu(rs)} + \frac{\mu'((1-r)s) \cdot -s}{\mu((1-r)s)} - \frac{K'(r)}{K(r)} - \frac{K'(1-r)}{K(1-r)}
\]
Setting $r=0$ and collecting constants gives
\[
&0
= \frac{\mu'(0)}{\mu(0)} s - \frac{\mu'(s) s}{\mu(s)} + C_2
= C_1 s - \frac{\mu'(s) s}{\mu(s)} + C_2
\\\implies & \frac{\mu'(s)}{\mu(s)} s = C_1 s + C_2
\\\implies & \frac{\mu'(s)}{\mu(s)} = C_1 + \frac{C_2}{s}
\\\implies & \ln\paren{\mu(s)}' = C_1 + \frac{C_2}{s}
\\\implies & \ln\paren{\mu(s)} = C_1 s + C_2 \ln\abs{s} + C_3
\\\implies & \mu(s) \propto s^C_2 \exp(C_1 s) = s^{\alpha - 1} \exp(-\lambda s)
\]

Beta:

As the assumption is that $\nu$ is independent of $s$, we have
\[
\nu(r)
&\propto \mu(r s) \mu((1-r)s)
\\&\propto (rs)^{\alpha - 1}\exp(-\lambda rs) ((1-r)s)^{\alpha - 1} \exp(-\lambda (1-r)s)
\\&\propto r^{\alpha - 1} (1-r)^{\alpha - 1} \exp(-\lambda (r + 1 - r)s)
\\&= r^{\alpha - 1} (1-r)^{\alpha - 1} \exp(-\lambda s)
\\&\propto r^{\alpha - 1} (1-r)^{\alpha - 1}
\]
which means $\nu$ is Beta distributed with both parameters equal to $\alpha$.


\section{Stationary for wealth dist}

As the wealth distribution model is takien analogous to a continuous random walk, jumping after an $\operatorname*{Exp}(1)$ waiting time, we have that
\[
Lf(r)
= \int \paren[\big]{f(T_{\lambda, \epsilon(r)}) - f(r) } \diff \nu(\epsilon)
= \int \paren[\big]{f(\lambda r + (1 - \lambda)\epsilon) - f(r) } \diff \nu(\epsilon)
\]
must be stationary under iteration of the jump process, i.e. the stationary measure is given by the one that stays invariant under iteration $R_{n+1} = T_{\lambda, \epsilon(R_n)}$. Writing out we have
\[
R_{n+1} = \lambda r_n + (1-\lambda)\epsilon_n
\\= \lambda^{n+1} + (1-\lambda)\sum_{k=0}^n \lambda^{k} \epsilon_{n-k}
\\\overset{\diff}{=} \lambda^{n+1} + (1-\lambda)\sum_{k=0}^n \lambda^{k} \epsilon_{k}
\to
 (1-\lambda)\sum_{k=0}^{\infty} \lambda^{k} \epsilon_{k} \eqdef \epsilon_{\infty}^{\lambda}
\]
which gives us the unique stationary distribution$\nu_{\infty}^{\lambda}$ for $L$.



\section{Duality}
Duality of Markov chains is a concept used to relate two Markov processes with each other. It is especially useful in relating a continuous state space Markov process with a discrete state space one, preserving information but simplifying analysis, especially in the context of simulations.

The following theorem is adapted from \cite{barbour2000transition}.

Let $(X_t)_t$ be a Markov process with Markov generator $L$ on some continuous state space $\Omega$, with some invariant measure $\mu$. Let $f_{\eta} \colon \Omega \to \Omega$, for $\eta$ from a discrete state space $\Omega'$. Assume
\[
L f_{\eta} = \sum_{\zeta} r(\eta, \zeta) f_{\zeta}
\]
holds for rates $r(\eta, \zeta) \ge 0$ if $\eta \ne \zeta$ and $r(\eta, \eta) \le 0$ for all $\eta$.  Define a generator $Q$ on $\Omega'$ by
\[
q(\eta, \zeta) \defeq r(\eta, \zeta)\frac{\mu(f_{\zeta})}{\mu(f_{\eta})}.
\]
Note that these indeed are rates as
\[
\sum_{\zeta} q(\eta, \zeta) = \frac{1}{\mu(f_{\eta})} \sum_{\zeta} r(\eta, \zeta)\mu(f_{\zeta}) = \frac{1}{\mu(f_{\eta})} \mu\paren*{\sum_{\zeta} r(\eta, \zeta) f_{\zeta}} = \frac{1}{\mu(f_{\eta})} \mu(L f_{\eta}) = 0.
\]
Let $(X_t')_t$ be the associated Markov process. We then have duality with the duality function
\[
D(\eta, x) = \frac{f_{\eta}}{\mu(f_{\eta})}
\]
as
\[
LD(\eta, \cdot)(x)
= \frac{f_{\eta}(x)}{\mu(f_{\eta})}
= \sum_{\zeta} \frac{r(\eta, \zeta)}{\mu(f_{\eta})} f_{\zeta}(x)
= \sum_{\zeta} \frac{q(\eta, \zeta)}{\mu(f_{\zeta})} f_{\zeta}(x)
= \sum_{\zeta} q(\eta, \zeta) D(\zeta, x)
= Q D(\cdot, x)(\eta)
\]
which implies
$\exp(tL) D(\eta, \cdot)(x) = \exp(tQ) D(\cdot, x)(\eta)$
% (the proof of this detail is technical, see \cite{voss2011equivalence} for a proof)
, which in turn implies (by definition) % https://www.researchgate.net/profile/Anja_Voss-Boehme/publication/261811929_On_the_Equivalence_Between_Liggett_Duality_of_Markov_Processes_and_the_Duality_Relation_Between_Their_Generators/links/00b7d5358d931ad3c6000000/On-the-Equivalence-Between-Liggett-Duality-of-Markov-Processes-and-the-Duality-Relation-Between-Their-Generators.pdf
\[
\ye{X}{D(\eta, X_t}
= \ye{\eta}{D(\eta_t, x}.
\]

In the case of $a(x, y) = -\alpha(x - y)$ we have
\[
\psi(r)
&\propto \frac{1}{r(1-r)} \exp\paren*{ \int \frac{-\alpha(r - (1-r)) s}{r(1-r)s} }
\\&= \frac{1}{r(1-r)} \exp\paren*{ \alpha \int \frac{1 - 2r}{r(1-r)} }
\\&= \frac{1}{r(1-r)} \exp\paren*{ \alpha (\ln(1-r) + \ln(r)) }
\\&= r^{\alpha - 1} (1-r)^{\alpha - 1}
\]



One can wonder whether there is a correspondence between duality of Markov processes and duality in the sense of dual spaces and adjoints of operators -- there happens to be one, explored in \cite{jansen2014notion}. %\cite{2012arXiv1210.7193J}. % https://arxiv.org/pdf/1210.7193.pdf




\section{Diffusion}

We have
\[
\partial_x
= \frac{\partial r}{\partial x}\partial_r + \frac{\partial s}{\partial x}\partial_s
= \frac{s - x}{s^2}\partial_r + \partial_s
\\\partial_y
= \frac{\partial r}{\partial y}\partial_r + \frac{\partial s}{\partial y}\partial_s
= \frac{0 - x}{s^2}\partial_r + \partial_s
\]
and hence
\[
\partial_x - \partial_y
= \frac{s - x + x}{x^2}\partial_r
= \frac{1}{s}\partial_r
\]
This gives
\[
(L_s \cdot)(r)
= \mu(r) \partial_r + \frac{\sigma(r)^2}{2} \partial_r^2
= \mu(r) \partial_r + h(r) \partial_r^2
\]
% http://www.math.wisc.edu/~shottovy/NumPDEreport.pdf
% https://www.math.nyu.edu/faculty/goodman/teaching/StochCalc2013/notes/Week9.pdf

Let $\mu(r) = \frac{a(rs, (1-r)s)}{s}, h(r) = \frac{\sigma(r)^2}{2}=(1-r)r$.

% Probability density should go to zero approaching the boundary points (in this case $0, 1$) 

The following is adapted from \cite{jonathangoodman2013}. To calculate the stationanry measure, one can consider duality pairings (in the functional setting) between functions $f \in D(L)$ and probability measures $\mu$ on the underlying state space. Suppose that $\mu$ is absolutely continuous, i.e. $\mu(\diff r) = \psi(r) \diff r$. The duality pairing is then between functions $f$ and densities $\psi$. We have
\[
0 = \angles{\psi, Lf} = \angles{L^* \psi, f}.
\]
% The density corresponding to the stationary distribution of a process with generator
Using the generator after the coordinate transform we have
\[
\angles{\psi, Lf}
&= \int \paren{\mu(r) \partial_r + h(r) \partial_r^2} f(r) \psi(r) \diff r
\\&= \int \mu(r) \psi(r) \partial_r \brackets{f(r)} \diff r + \int h(r) \psi(r) \partial_r^2\brackets{f(r)} \diff r
\\&= -\int \partial_r \brackets{\mu(r) \psi(r)} f(r) \diff r + (-1)^2\int \partial_r^2 \brackets{h(r) \psi(r)} f(r) \diff r
\\&= \int \paren{ -\partial_r \brackets{\mu(r) \psi(r)} + \partial_r^2 \brackets{h(r) \psi(r)}} f(r) \diff r
= \angles{L^* \psi, f}
\]
Hence invariance of $\mu$, i.e. $0 = \mu(Lf) = \angles{\psi, Lf} = \angles{L^* \psi, f}$ for all $f$ implies the Kolmogorov forward equation or Fokker--Planck equation:
%, see also the Feynman--Kac equation corresponding to the backward equation):
\[
0 = -\brackets{g(r) \psi(r)}' + \brackets{h(r) \psi(r)}''
\\\implies g(r) \psi(r) = \brackets{h(r) \psi(r)}'
\]
where the constant is zero due to the absorbing boundary conditions; for any $r'$ outside of the support we have $0 = g(r') \psi(r') = \brackets{g(r') \psi(r')}' + C = 0 + C$. Now setting $y(r) = h(r) \psi(r)$ we get
\[
y' = \frac{\mu}{h} y \implies (\ln(y))' = \frac{\mu}{h}
\\ \implies y(r) = C \exp\paren*{ \int \frac{\mu(r)}{h(r)}\diff r }
\\ \implies \psi(r) = \frac{C}{h(r)} \exp\paren{ \int \frac{\mu(r)}{h(r)}\diff r }
\]
i.e.
\[
\psi(r) = \frac{C}{r(1-r)} \exp\paren*{ \int \frac{a(rs, (1-r)s)}{r(1-r)s}\diff r }.
\]

Now as the stationary measure $\mu$ is unique and the same as the stationary measure of the distribution model in the case $\lambda = 0$, we have for the limit of starting the semigroup at an arbitrary point $(x, y)$ that
\[
\lim_{t\to\infty} \ye{x, y}{f(X_t, Y_t)}
= \mu(Pf)
\]
and hence for the generator of the original process $\mathcal{L}$
\[
\mathcal{L}f(x, y)
= Pf(x, y) - f(x, y)
= \lim_{t\to\infty} P_t f(x, y) - f(x, y)
= \lim_{t\to\infty} \exp(tL) f(x, y) - f(x, y).
\]


\section[nagents]{The wealth distribution model for $N$ agents}
Consider set of agents $\brackets{N} \defeq \set{1, 2, \hdots, N}$ and a symmetric random walk $\paren{\mathcal{X}_t}_t$ jumping at rate one according to transition probabilities $p(i,j)$, and (vector) process $x(t)$ of wealth of agents

Let $T_{\lambda, \epsilon}^{i, j}(x)$ for $x \in [0, \infty)^{\brackets{N}}$ such that
\[
T_{\lambda, \epsilon}^{i, j}(x)_k = T_{\lambda, \epsilon}(x_i, x_j)_k
% equal to \lambda x + (1- \lambda) \cdot (\epsilon OR 1-\epsilon) (x+y)
T_{\lambda, \epsilon}^{i, j}(x)_k = T_{\lambda, \epsilon}(x_i, x_j)_k
\]

The generator then is given by
\[
\mathcal{L}f(x) = \sum_{i, j} 2 p(i, j) \int \paren*{f(T_{\lambda, \epsilon}^{i, j}(x)) - f(x)} \nu(x+y, \diff \epsilon)
\]

% Now as $\exp(tL) = S_t f(x) = \ye{x}{f(X_t)}$ and $\frac{\partial}{\partial_t} S_t = A S_t$ we have with $\psi() \defeq S_t f_i(x)$ that ...
Define $\bar{f}_x(i) = f_i(x)$, a function $\itern \to \rr$. This is a function the generator $A$ of the continuous random walk $(\mathcal{X}_t)_t$ works on. Note that the fact that this is the generator of a finite state-space Markov process was established in equation (17) section 2.2 of \cite{frankredig2014}.

We have
\[
Lf_i(x) = (1 - \lambda) A \bar{f}_x(i)
\implies &
\exp(tL)f_i(x) = \exp((1 - \lambda)t A) \bar{f}_x (i)
% \\\implies &
% S_t f_i(x) = P_{(1 - \lambda)t} \bar{f}_x(i)
\\\implies &
\ye{x}{f_i(x(t))} = \mathcal{E}_i\brackets{\bar{f}_x(\mathcal{X}_{(1-\lambda)t})}
\\\implies &
\ye{x}{x_i(t)} = \mathcal{E}_i\brackets{x_{\mathcal{X}_{(1-\lambda)t}}}
\]
% Note we again used the technical result from \cite{voss2011equivalence}.


\section*{Further reading}
An overview of other models for wealth and income distributions can be found in \cite{chakrabarti2013econophysics}. Further work by one of the authors of \cite{cirillo2014duality} in the econophysics of income and wealth distributions is for example found in \cite{redig2015multilinearity,  redig2017generalized, 2016JSP...163...92V}.




\section*{Conclusion}
For the wealth distribution model discussed, we found that in the case of nonzero propensity i.e. if the agents save any money there do not exist any product stationary measures.



\printbibliography

\end{document}
