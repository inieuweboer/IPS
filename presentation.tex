\documentclass[9pt, handout]{beamer}

\usepackage{amsmath}
\usepackage{amsfonts}
\usepackage{amssymb}
\usepackage{mathtools}
\usepackage{amsthm}
\usepackage{bbm}

\input{structure/aliases}

\usetheme[progressbar=frametitle]{metropolis}
\usepackage{appendixnumberbeamer}

\usepackage{booktabs}
% \usepackage[scale=2]{ccicons}

% \usepackage{pgfplots}% \usepgfplotslibrary{dateplot}

\usepackage{xspace}
\newcommand{\themename}{\textbf{\textsc{metropolis}}\xspace}

\usepackage{color,soul}

\usepackage{caption}
% \usepackage{pgffor}

\usepackage{csquotes}
\usepackage{ragged2e}
\usepackage[
    % niks qua stijl enzo lijkt het beste
    backend=biber,
    maxbibnames=99,
]{biblatex}
\addbibresource{refs.bib}
\setbeamertemplate{bibliography item}{\insertbiblabel}

\usepackage{hyperref}


% \DeclarePairedDelimiter\floor{\lfloor}{\rfloor}

\metroset{block=fill}


% Authors and title
\newcommand*{\xauthor}{Ismani Nieuweboer}
\newcommand*{\xauthortiny}{I. Nieuweboer}
\newcommand*{\xtitle}{Wealth distribution models\xspace}
\newcommand*{\xsubtitle}{Duality and stationary distributions\xspace}
% \newcommand*{\xsubtitle}{\xspace}



\title{\xtitle}
\subtitle{\xsubtitle}
\date{2019-01-23}
% \date{}
\author{\xauthor}
\institute{University of Amsterdam}
% \titlegraphic{\hfill\includegraphics[height=1.5cm]{uvalogoofzo.pdf}}
% \setbeamertemplate{frame footer}{\xtitle -- \xauthortiny}
% \setbeamercolor{frame footer}{bg=beamer@blendedblue}

\begin{document}

\maketitle

% \begin{frame}{Table of contents}
%   \setbeamertemplate{section in toc}[sections numbered]
%   \tableofcontents[hideallsubsections]
% \end{frame}

% \section{Introduction}

\begin{frame}{Introduction}
    \begin{itemize}
    \item Econophysics is a research field applying methods and ideas of statistical physics to economics
    \item Interest for economics from people originating from mathematics and physics is not new; for example Daniel Bernoulli, Jan Tinbergen -- see also \authorfootcite{yakovenko2009colloquium} for more examples
    \item This presentation will focus on concepts and results from \authorfootcite{cirillo2014duality}
    \end{itemize}
\end{frame}



\begin{frame}{Mathematical models and scientific method}
\begin{itemize}
    \item Model is mathematical 
    \item Need to take care with interpretations and drawing conclusions about the real world -- empirical research is needed for this
    \item Need to take care especially due to the evident link between politics and economics
    % \item Also need to take care about language one uses -- one may still end up unconsciously make light political statements (?)
    % \item Hereby a warning to the listener to make sure the speaker does not cross these boundaries
\end{itemize}
\end{frame}


\begin{frame}{Concepts}
\begin{itemize}
    \item Money: Medium of exchange, store of value, unit of account (including debt)
    \item Wealth: Total of assets measured in monetary value
    \item Income: sum of all earnings over given period of time
    \item Propensity to save: 
\end{itemize}
For example, one theoretical model in \cite{yakovenko2009colloquium} argues money is analogous to energy and income is analogous to power (energy per unit of time).
\end{frame}


\begin{frame}{Product measures}
\begin{itemize}
    \item
\end{itemize}
\end{frame}


\begin{frame}{Duality}
\begin{theorem}
\end{theorem}
\begin{itemize}
    \item
\end{itemize}
\end{frame}


\begin{frame}{Diffusion process}
\begin{itemize}
    \item \authorcite{yakovenko2009colloquium} also mentions the Fokker--Planck equation in setting for income distributions
\end{itemize}
\end{frame}




\begin{frame}{Wrapping up}
  \begin{block}{Summary}
    \begin{itemize}
    \item A class of energy and wealth distribution models have been described
    \item With nonzero saving propensity these distributions do not have invariant measures
    \item A class of associated diffusion processes exist
    \item A generalized model for $N$ agents is explored
    \end{itemize}
  \end{block}
  \begin{block}{Further reading}
    \begin{itemize}
    \item \authorcite{chakrabarti2013econophysics} has an overview of econophysical models of income and wealth distributions
    \item Further work building on paper discussed can be found through the citations: \url{https://scholar.google.com/scholar?cites=17345843713482550092}
    \end{itemize}
  \end{block}
  \pause
  \begin{center}
%   \vspace{1em}
  \huge{
  Thank you for your attention!
  }
  \end{center}
\end{frame}

\begin{frame}[allowframebreaks]{References}

%   \bibliography{ijcai17}
%   \bibliographystyle{abbrv}
  \printbibliography
\end{frame}

\end{document}
