\documentclass[9pt, handout]{beamer}

\usepackage{amsmath}
\usepackage{amsfonts}
\usepackage{amssymb}
\usepackage{mathtools}
\usepackage{amsthm}
\usepackage{bbm}

% \usepackage{amsmath}
% \usepackage{amsfonts}
% \usepackage{amssymb}
% \usepackage{mathtools}
% \usepackage{amsthm}
% \usepackage{bbm}
% % \usepackage{algpseudocode}
% \usepackage{interval}

\newcommand*{\citeauthorfoot}[1]{\citeauthor{#1}\footfullcite{#1}} 


\def\[#1\]{\begin{align*}#1 \end{align*}}  % Can do newlines in \[ \] env

\renewcommand*{\Re}{\operatorname{Re}}
\renewcommand*{\Im}{\operatorname{Im}}
\newcommand*{\ff}{\mathbb{F}}
\newcommand*{\zz}{\mathbb{Z}}
\newcommand*{\qq}{\mathbb{Q}}
\newcommand*{\rr}{\mathbb{R}}
\newcommand*{\cc}{\mathbb{C}}

\newcommand*\diff{\mathop{}\!\mathrm{d}}
\newcommand*\Diff[1]{\mathop{}\!\mathrm{d^#1}}\newcommand*{\defeq}{\vcentcolon=}
% \newcommand*{\defeq}{\vcentcolon=}
\newcommand*{\eqdef}{=\vcentcolon}

\newcommand*{\indic}{\mathbbm{1}}
\newcommand*{\Prob}{\mathbb{P}}
\newcommand*{\EE}{\mathbb{E}}

\newcommand*{\property}{\mathcal{P}}

\newcommand*{\Id}{\operatorname*{Id}}

\newcommand*{\LL}{\mathcal{L}}
\newcommand*{\GG}{\mathcal{G}}
\newcommand*{\MM}{\mathcal{M}}

\newcommand*{\normal}{\mathcal{N}}

\newcommand*{\Xx}{\mathbf{x}}
\newcommand*{\Yy}{\mathbf{y}}
\newcommand*{\Zz}{\mathbf{z}}

\newcommand*{\lb}{\log_2}

\DeclarePairedDelimiter{\ceil}{\lceil}{\rceil}
\DeclarePairedDelimiter{\floor}{\lfloor}{\rfloor}
\DeclarePairedDelimiter{\setx}{\lbrace}{\rbrace}
\DeclarePairedDelimiterXPP{\set}[1]{}{\lbrace}{\rbrace}{}{
  \renewcommand\given{\nonscript\:\delimsize\colon\nonscript\:\mathopen{}} #1
}
\DeclarePairedDelimiter{\brackets}{[}{]}
\DeclarePairedDelimiterX{\paren}[1]{(}{)}{#1}
%\DeclarePairedDelimiterX{\paren}[1]{(}{)}{
%  \ifblank{#1}{\:\cdot\:}{#1}
%}
\DeclarePairedDelimiterX{\abs}[1]{\lvert}{\rvert}{
  \ifblank{#1}{\:\cdot\:}{#1}
}
\DeclarePairedDelimiterX{\inp}[2]{\langle}{\rangle}{
  \ifblank{#1}{\:\cdot\:}{#1}
  \mathop{}\mathclose{}
  \delimsize\vert
  \mathopen{}\mathop{}
  \ifblank{#2}{\:\cdot\:}{#2}
}
\DeclarePairedDelimiterX{\norm}[1]{\lVert}{\rVert}{
  \ifblank{#1}{\:\cdot\:}{#1}
}
\DeclarePairedDelimiterXPP{\dist}[3]{d_{#1}}{(}{)}{}{#2, #3}

\providecommand\given{}
\DeclarePairedDelimiterXPP{\xprob}[1]{\Prob}{[}{]}{}{
  \renewcommand\given{\nonscript\:\delimsize\vert\nonscript\:\mathopen{}} #1
}
\DeclarePairedDelimiterXPP{\yprob}[2]{\Prob_{#1}}{[}{]}{}{
  \renewcommand\given{\nonscript\:\delimsize\vert\nonscript\:\mathopen{}} #2
}
\DeclarePairedDelimiterXPP{\uprob}[2]{\operatorname*{\Prob}_{\substack{#1}}}{[}{]}{}{
  \renewcommand\given{\nonscript\:\delimsize\vert\nonscript\:\mathopen{}} #2
}
\DeclarePairedDelimiterXPP{\xe}[1]{\EE}{[}{]}{}{#1}
\DeclarePairedDelimiterXPP{\xv}[1]{\mathbb{V}\mathrm{ar}}{[}{]}{}{#1}
\DeclarePairedDelimiterXPP{\xvar}[1]{\mathrm{Var}}{[}{]}{}{#1}

\DeclarePairedDelimiterXPP{\ye}[2]{\EE_{#1}}{[}{]}{}{#2}
\DeclarePairedDelimiterXPP{\ue}[2]{\operatorname*{\EE}_{\substack{#1}}}{[}{]}{}{#2}
\DeclarePairedDelimiterXPP{\xI}[1]{\indic}{[}{]}{}{#1}
\DeclarePairedDelimiterXPP{\yI}[2]{\indic_{#1}}{[}{]}{}{#2}



\newcommand*{\itern}{\brackets{n}}
\DeclarePairedDelimiter{\dual}{\bigl(}{\bigr)'}

\DeclarePairedDelimiterXPP{\xLn}[1]{\operatorname{\mathrm{ln}}}{(}{)}{}{#1}
%\DeclarePairedDelimiterXPP{\deg}[1]{\operatorname{\mathrm{deg}}}{(}{)}{}{#1}
\DeclarePairedDelimiterXPP{\Dim}[1]{\operatorname{\mathrm{dim}}}{(}{)}{}{#1}
\DeclarePairedDelimiterXPP{\Sp}[1]{\operatorname{\mathrm{Sp}}}{(}{)}{}{#1}
\DeclarePairedDelimiterXPP{\Ker}[1]{\operatorname{\mathrm{Ker}}}{(}{)}{}{#1}


\DeclarePairedDelimiterXPP{\fw}[2]{\mathbf{W}^{#1}}{[}{]}{}{#2}

\DeclarePairedDelimiterXPP{\Stab}[2]{\mathbf{Stab}_{#1}}{[}{]}{}{#2}
\DeclarePairedDelimiterXPP{\Spec}[2]{\mathbf{Spec}_{#1}}{[}{]}{}{#2}


% \newcommand*{\Ftwo}{\set{\pm 1}}


\newcommand*{\inftoinf}{\brackets{-\infty .. \infty}}
\newcommand*{\ivdlm}{,}

% \newcommand*{\holder}{H\"{o}lder }
% \newcommand*{\hct}{\(\paren[]{p, q, \rho}\)}

% obsolete
%\DeclarePairedDelimiterXPP{\xProb}[1]{\Prob}{[}{]}{}{#1}
%\DeclarePairedDelimiter{\abs}{\lvert}{\rvert}
%\DeclarePairedDelimiter{\norm}{\lVert}{\rVert}
%\DeclarePairedDelimiter{\set}{\lbrace}{\rbrace}
%\DeclarePairedDelimiterX{\zE}[1]{\operatorname*{\E}_{\substack{#1}}}{}{}
%\DeclarePairedDelimiter{\fw}{[}{]}



% Mooiere lege verzameling
\let\oldemptyset\emptyset
\let\emptyset\varnothing
% \let\oldphi\phi
% \let\phi\varphi


\usetheme[progressbar=frametitle]{metropolis}
\usepackage{appendixnumberbeamer}

\usepackage{booktabs}
% \usepackage[scale=2]{ccicons}

% \usepackage{pgfplots}% \usepgfplotslibrary{dateplot}

\usepackage{xspace}
\newcommand{\themename}{\textbf{\textsc{metropolis}}\xspace}

\usepackage{color,soul}

\usepackage{caption}
% \usepackage{pgffor}

\usepackage{csquotes}
\usepackage{ragged2e}
\usepackage[
    % niks qua stijl enzo lijkt het beste
    backend=biber,
    maxbibnames=99,
]{biblatex}
\addbibresource{refs.bib}
\setbeamertemplate{bibliography item}{\insertbiblabel}

\usepackage{hyperref}


% \DeclarePairedDelimiter\floor{\lfloor}{\rfloor}

\metroset{block=fill}


% Authors and title
\newcommand*{\xauthor}{Ismani Nieuweboer}
\newcommand*{\xauthortiny}{I. Nieuweboer}
\newcommand*{\xtitle}{Wealth distribution models\xspace}
\newcommand*{\xsubtitle}{Duality and stationary distributions\xspace}
% \newcommand*{\xsubtitle}{\xspace}



\title{\xtitle}
\subtitle{\xsubtitle}
\date{2019-01-23}
% \date{}
\author{\xauthor}
\institute{University of Amsterdam}
% \titlegraphic{\hfill\includegraphics[height=1.5cm]{uvalogoofzo.pdf}}
% \setbeamertemplate{frame footer}{\xtitle -- \xauthortiny}
% \setbeamercolor{frame footer}{bg=beamer@blendedblue}

\begin{document}

\maketitle

% \begin{frame}{Table of contents}
%   \setbeamertemplate{section in toc}[sections numbered]
%   \tableofcontents[hideallsubsections]
% \end{frame}

% \section{Introduction}

\begin{frame}{Introduction}
    \begin{itemize}
    \item Econophysics is a research field applying methods and ideas of statistical physics to economics
    \item Interest for economics from people originating from mathematics and physics is not new; for example Daniel Bernoulli, Jan Tinbergen -- see also \authorfootcite{yakovenko2009colloquium} for more examples
    \item This presentation will focus on concepts and results from \authorfootcite{cirillo2014duality}
    \end{itemize}
\end{frame}



\begin{frame}{Mathematical models and scientific method}
\begin{itemize}
    \item Model is mathematical 
    \item Need to take care with interpretations and drawing conclusions about the real world -- empirical research is needed for this
    \item Need to take care especially due to the evident link between politics and economics
    % \item Also need to take care about language one uses -- one may still end up unconsciously make light political statements (?)
    % \item Hereby a warning to the listener to make sure the speaker does not cross these boundaries
\end{itemize}
\end{frame}


\begin{frame}{Concepts}
\begin{itemize}
    \item Money: Medium of exchange, store of value, unit of account (including debt)
    \item Wealth: Total of assets measured in monetary value
    \item Income: sum of all earnings over given period of time
    \item Propensity to save: 
\end{itemize}
For example, one theoretical model in \cite{yakovenko2009colloquium} argues money is analogous to energy and income is analogous to power (energy per unit of time).
\end{frame}


\begin{frame}{Product measures}
\begin{itemize}
    \item
\end{itemize}
\end{frame}


\begin{frame}{Duality}
\begin{theorem}
\end{theorem}
\begin{itemize}
    \item
\end{itemize}
\end{frame}


\begin{frame}{Diffusion process}
\begin{itemize}
    \item \authorcite{yakovenko2009colloquium} also mentions the Fokker--Planck equation in setting for income distributions
\end{itemize}
\end{frame}




\begin{frame}{Wrapping up}
  \begin{block}{Summary}
    \begin{itemize}
    \item A class of energy and wealth distribution models have been described
    \item With nonzero saving propensity these distributions do not have invariant measures
    \item A class of associated diffusion processes exist
    \item A generalized model for $N$ agents is explored
    \end{itemize}
  \end{block}
  \begin{block}{Further reading}
    \begin{itemize}
    \item \authorcite{chakrabarti2013econophysics} has an overview of econophysical models of income and wealth distributions
    \item Further work building on paper discussed can be found through the citations: \url{https://scholar.google.com/scholar?cites=17345843713482550092}
    \end{itemize}
  \end{block}
  \pause
  \begin{center}
%   \vspace{1em}
  \huge{
  Thank you for your attention!
  }
  \end{center}
\end{frame}

\begin{frame}[allowframebreaks]{References}

%   \bibliography{ijcai17}
%   \bibliographystyle{abbrv}
  \printbibliography
\end{frame}

\end{document}
