%%%%%
% Author:
% Ismani Nieuweboer
%   ismani.nieuweboer@student.uva.nl
%   10502815
%
% Compile from this file to create the pdf.
% Last checked working on overleaf.com
%%%%%

\documentclass{article}
\usepackage{structure/mystyle}



\title{Econophysics\\
\normalsize{Summary}
}
\author{Ismani Nieuweboer}
% \date{September 2018}

\begin{document}

\maketitle

\url{https://arxiv.org/abs/0905.1518}
\url{https://arxiv.org/abs/1309.3916}

Colloquium article Victor Yakovenko: ``Colloquium: Statistical mechanics of money, wealth and income''.

Money is conserved. Wealth is money + assets * asset prices. Income distribution follows.

Money temperature

In a system/model where the total amount of money is conserved and the individual amount is lower bounded per person, and constant random transfers happen, the Boltzmann-Gibbs distribution (or exponential distribution?) (of energy) appears. The money temperature rises (page 8).

In a system/model where the total amount of money is conserved and the total amount of debt is bounded, two exponential distributions appear (page 8, fig 3).

Proportional money transfer gives Gamma-like dist

Page 10: Additive vs multiplicative model.

Exp is additive (corresponds to lower class), power law is multiplicative (corresponds to upper class).

Page 18: Silva, Yakovenko -- Fokker Planck equation
\[
\]
gives ??
\[
\frac{\partial P}{\partial t}
= \frac{\partial}{ \partial r} (AP + \frac{\partial (BP)}{ \partial r}
\]
with $A = - \frac{\xE{\Delta r}}{\Delta t}$, $B = \frac{\xE{(\Delta r)^2}}{2 \Delta t}$ % (remember from queueing theory
first and second moment of income changes per unit time.

Questions
\begin{itemize}
    \item Do we have cvg of Boltzmann-Gibbs to Exp in some sense of parameter choice?
    \item Page 8 makes classical assumption about money multiplier, but Positive money e.a. show that this is not how financial system in UK and Europe work. What happens with model?
\end{itemize}


\end{document}
