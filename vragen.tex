lijst van vragen/todos

% invariant measures beter begrijpen
% duality beter begrijpen
harmonic fns beter begrijpen

Diffusion process is a Markov process with cont sample paths for which Kolmogorov forward equation is Fokker Planck equation.

inclusion vs exclusion: Inclusion without max amount particles, exclusion with max amt particles at sites.

% p16 limiting dist - semigroups van aparte generators bekijken!

% technical result used in duality and n agents
% Use Taylor series! Otherwise use commutation and solution of matrix ODE.

% Bij thm over duality, is $f_n(r) = r^n$ strikt nodig? What about context van K(., ., .)
% Nee, alleen f >= 0

% p5 (grand) canonical?
% canonical: s vast
% grand canonical: s niet vast en lineaire combinatie

% p8 unique, ergodic ,dual chain irreducible -- compact set heeft invariant measure

% page 17 eerste stap in vergelijking met factor 2 ding
% twee keer hetzelfde lol



meh/done

page 5 rmk 2.2 (1-epsilon) ipv epsilon?

page 9 rmk 3.1 hoe leid je pareto dist af?

% page 9 eq 15
% Kom je hierop dmv $\nu(Lf) = 0$? % Komt omdat model analoog is aan random walk, met Exp(1) jumpen.

% page 10
% wat betekent een distribution over $s$? Aangezien deze conserved is? %Aansluitend, is hoeveelheid wealth $s=x+y$ conserved over tijd? (gok: ja). Wealth distribution afhankelijk van wealth als in progressief oid


% page 12 welke theorem gaat het om in ref [1], is deze nodig?

% page 12 hoe volgt eq (24)? geeft exp van generator nemen tegelijk met functie $D$ de gewenste verwachting?
% Dit volgt met technisch resultaat van Voss-Boehme, Schenk et al

% page 13  hoe volgt $\int x^n \diff \mu(x) = \frac{\Gamma(b)}{\Gamma(b+n)}$ ?
% nth moment gamma distribution ? This has extra factor

% page 13 hoe eq (28)?

% p13 kan ik $\lambda = 0$ nemen in 4.2 en andere resultaat krijgen?
% antw: nee, want de rates worden zero, i.e. discreet proces is triviaal/deterministisch

% wealth distribution vs energy distribution? i.e. 4.1 vs 4.2

% page 15 r and s swapped eqs (36) and (37)? Volgens berekening eq (36) fout.
% page 15 minteken in een voor laatste eq
% page 15 hoe volgt laatste eq uit (37)?
% page 16 is $\nu(s, r) = \psi(r)/C$ ? beetje getover dit

